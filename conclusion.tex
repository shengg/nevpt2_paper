
\section{Conclusion}

In this work, we developed a general way to compute different order RDM for DMRG wave function, which can be used by many dynamic correlation 
methods. NEVPT2 is a intruder state free second order multireference perturbation theoy. Through up to 4th order RDM, we combined DMRG and NEVPT2 
to discribe static and dynamic correlations at the same time, forming DMRG-NEVPT2 method. It made the calculations complex molecules with large active spaces possible. 

To demonstrate the capability of DMRG-NEVPT2 method, we used this method to calculate the potential energy curve of chromium dimer and excited 
states of Poly(p-phenylene vinylene). For chromium dimer, the extended active space included $4d$ orbitals and the so called ``double d'' effect 
were considered. It gave quantatively corrected potential energy curves, demonstrating that correlation from second shell orbitals were essential for chromium dimer.
For light emitting conjugated polymers, Poly(p-phenylene vinylene), excitation energy of low-lying excited states were computed through DMRG-NEVPT2 
with different orbitals in active space. The local orbitals calculation gave results converged with bond dimension much better than canonical 
orbitals. And the accuracy from local orbital DMRG-NEVPT2 is about $0.6meV$, relative to standard NEVPT2. While the environments were not included 
in the calculations, the energy order of excited states were reasonable.

Based on above two examples, DMRG-NEVPT2 is a intruder free perturbation methods and it can be used for the molecule where large active spaces are needed.