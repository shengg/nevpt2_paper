
\section{Conclusion}

In this work we described an efficient and general algorithm to compute high order RDM's from a DMRG wave function.
These RDM's may be used as the starting point for many kinds of internally contracted dynamic correlation methods.
Using up to the 4th order RDM, we combined DMRG and strongly contracted NEVPT2 to obtain
an intruder state free second order multireference perturbation theory that can be used with large active spaces.

To demonstrate the capability of the DMRG-SC-NEVPT2 method, we calculated the potential energy curve of the chromium dimer and the excited 
states of PPV3. For the chromium dimer, the best extended active space included the so called ``double d'' shell.
Our obtained curve compares quite favourably with earlier calculations and tracks the experimental curve very closely, except
at very long bond-lengths where the experimental inversion may be suspect. We find that the semi-core orbitals
are much less important for an accurate description of this curve.
For PPV3, using an active space containing the full set of $\pi, \pi^*$ orbitals, we found that the dynamic correlation included through NEVPT2 allowed us to obtain
a correct ordering of the low-lying excited states, and the lowest excitation energy is in good agreement with
the experimental data. In summary, these examples paint an optimistic picture for the potential of applying the
DMRG-SC-NEVPT2 method developed here to a wide range of challenging problems.
