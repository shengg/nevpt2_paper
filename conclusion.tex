
\section{Conclusion}

In this work, we described an efficient and general algorithm to compute high order RDM's from a DMRG wave function.
These RDM's may be used as the starting point for many
different kinds of internally contracted dynamic correlation methods.
Using up to the 4th order RDM, we combined DMRG and strongly contracted NEVPT2 to obtain
an intruder state free second order multireference perturbation theory that can be used with large active spaces.

To demonstrate the capability of the DMRG-SC-NEVPT2 method, we calculated the potential energy curve of the chromium dimer and the excited 
states of poly(p-phenylene vinylene) (PPV). For the chromium dimer, the extended active space included $4d$ orbitals and the so called ``double d'' shell.
Our obtained curve compares quite favourably with earlier calculations and tracks the experimental curve very closely, except
at very bond-lengths where the experimental inversion may be suspect.
For PPV-3, using an active space containing the full set of $\pi, \pi^*$ orbitals, we found that the dynamic correlation included through NEVPT2 allowed us to obtain
a reasonable ordering of the low-lying excited states, and a lowest excitation energy in good agreement with earlier semi-empirical estimates, and
the experimental absorption result. In summary, these examples paint an optimistic picture for the potential of applying the
DMRG-SC-NEVPT2 methodology to a wide range of challenging problems.
