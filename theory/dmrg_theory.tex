
\subsection{DMRG wavefunction and optimization algorithm}

As with other wavefunction methods in quantum chemistry, DMRG is based on a wavefunction ansatz, namely the matrix product state (MPS).
An MPS is a non-linear wavefunction, built from a contraction of tensors for each orbital in the basis. If an MPS is
composed of tensors with a limited dimension (called the bond dimension, $M$) it explores only a physically motivated, but restricted, subset of the Hilbert space. 
By increasing $M$, the MPS ansatz will then converge to the full configuration interaction (FCI) result. The DMRG algorithm provides a combination of 
renormalization and truncation steps that efficiently find a variational and optimal MPS. 
In practice, two slightly different types of MPS are used in practical
DMRG calculations: the one-site and two-site MPS. 

The one-site MPS is defined as
\begin{equation}
  \ket{\Psi}= \sum_{\substack{n_1, n_2\cdot\cdot\cdot n_p\cdot\cdot\cdot n_k}} {\bf A}^{n_1}{\bf A}^{n_2}\cdot\cdot\cdot {\bf A}^{n_{p}} \cdot\cdot\cdot{\bf A}^{n_k}
\end{equation}
Here, $n_i$ is the occupacy of orbital $i$, one of $ \{\ket{}, \ket{\uparrow}, \ket{\downarrow}, \ket{\uparrow\downarrow}\}$, and $k$ is the number of orbitals. 
For a given $n_i$, ${\bf A}^{n_i}$ is  an $M\times M$ matrix, except for the first and last ones; the dimensions of ${\bf A}^{n_1}$ and ${\bf A}^{n_k}$ are $1\times M$ and $M\times 1$ respectively.
This ensures that for a given occupancy string $n_1n_2\cdots n_k$  the ${\bf A}^{n_1}{\bf A}^{n_2}\cdots{\bf A}^{n_k}$ product yields a scalar, 
namely, the coefficient of the determinant $\ket{n_1n_2\cdots n_k}$.

Similarly, the two-site MPS is defined as
\begin{equation}
  \ket{\Psi}= \sum_{\substack{n_1, n_2\cdot\cdot\cdot n_p\cdot\cdot\cdot n_k}} {\bf A}^{n_1}{\bf A}^{n_2}\cdot\cdot\cdot {\bf A}^{n_{p}n_{p+1}} \cdot\cdot\cdot{\bf A}^{n_k}
\end{equation}
The only difference with the one-site MPS, is that one of the tensors (${\bf A}^{n_p n_{p+1}}$) is associated with two sites at a time. The two-site MPS thus
has slightly more variational freedom than the one-site MPS, and further
exists in different non-equivalent forms, depending on which sites are chosen for the special two-site tensor.

%% This additional tensor, which
%% would require two MPS tensors at positions $p$, $p+1$ to have bond dimension $4 M$ rather than $M$ if written in the one-site form, is useful
%% during the numerical optimization of the MPS, as it provides information outside of the space strictly explored by MPS of bond dimension $M$ that
%% can be used to help improve the convergence, or, to extrapolate the energy as a function of the discarded weight in the renormalization.

In the DMRG sweep algorithm, $\bra{\Psi}H\ket{\Psi}/\braket{\Psi|\Psi}$ is variationally optimized, and  a single tensor ($\mathbf{A}^{n_i}$ in a one-site MPS or $\mathbf{A}^{n_i, n_{i+1}}$ in a  two-site MPS) is updated at site $i$ during the sweep. This update leads to solving an eigenvalue problem of an effective ``superblock'' Hamiltonian, 
which is a $4M^2\times 4M^2$ matrix. 
The multiplication between this effective Hamiltonian and a MPS is usually computed using $O(k^2)$ ``normal'' and ``complementary'' operators 
built from different sets (blocks) of the active orbitals. The cost of this eigenvalue problem is $O(k^2M^3)$ for a single tensor update, and $O(k^3M^3)$ 
for a whole sweep optimization. 

As discussed, a two-site MPS has more variational freedom than a one-site MPS at the site where the special tensor is placed. This allows 
the DMRG optimization to change quantum numbers, i.e. conserved local symmetries, during the sweep, and thus helps avoid local minima. Further,
the discarded weight associated with the renormalization step in a two-site MPS sweep optimization can be used to extrapolate away the residual 
error due to finite bond dimension $M$. 
However, as the two-site MPS wavefunction at each step of the sweep 
is slightly different (because the special tensor is moved along the sweep),
the two-site formalism does not provide a unique definition of the DMRG 
expectation values.
Thus, to evaluate the reduced density matrix, we complete the DMRG
sweep optimization in the one-site representation, to obtain a single 
DMRG wavefunction. This is
referred to as the ``two-site to one-site'' algorithm (a two-site MPS optimization followed by a one-site one)~\cite{olivares-amaya_ab-initio_2015}. 
From here on when we refer to an MPS, we will mean the one-site MPS.


%A mixed canonical form of MPS at site $p$ is 
%
%\begin{eqnarray}
%    \ket{\Psi_{MPS}}=&& \sum_{\substack{n_1\cdot\cdot\cdot n_p\cdot\cdot\cdot n_k\\l_1\cdot\cdot\cdot l_{p-1},r_p\cdot\cdot\cdot r_{k-1} }} L^{n_1}_{l_1}\cdot\cdot\cdot L^{n_{p-1}}_{l_{p-2},l_{p-1}} C^{n_p}_{l_{p-1},r_p}\nonumber\\
%&&  R^{n_{p+1}}_{r_p,r_{p+1}}\cdot\cdot\cdot R^{n_k}_{r_{k-1}}
%\times\ket{n_1\cdot\cdot\cdot n_p \cdot\cdot\cdot n_k}
%\end{eqnarray}
%
%$L^{n_i}_{l_{i-1},l_i}$ is for the sites on the left of p and $R^{n_i}_{r_{i-1},r_i}$ is for the sites on the right of $p$. $n_i$ is one of four single site states \{$\ket{0},\ket{\uparrow},\ket{\downarrow},\ket{\uparrow\downarrow}$\}. (In spin-adapted dmrg code, block, (citation here) only three states are necessary.). The dimension of bond index $l_i$ or $r_i$ (M) is the number of kept renormalized states. MPS is more accurate with a higher M. L and R site functions are canonical by following definition,
%
%\begin{equation}
%\sum_{n_q,l_{q-1}} L^{n_q}_{l_{q-1},l_q} L^{n_q}_{l_{q-1},l'_q} = \delta_{l_q,l'_q}
%\end{equation}
%
%\begin{equation}
%\sum_{n_q,l_q} R^{n_q}_{r_{q-1},r_q} R^{n_q}_{r'_{q-1},l_q} = \delta_{r_{q-1},r'_{q-1}}
%\end{equation}
%
%Combine L site functions and occupation number basis for the left block, which is formed by the sites on the left of p, we get
%
%\begin{eqnarray}
%\ket{l_{p-1}} = &&\sum_{\substack{n_1,n_2,..n_{p-1}\\l_1,l2,...,l_{p-2}}} L^{n_1}_{l_1}L^{n_2}_{l_1,l_2}\cdot\cdot\cdot L^{n_{p-1}}_{l_{p-2},l_{p-1}}\nonumber\\
%&&\times \ket{n_1,n_2,..,n_{p-1}}
%\end{eqnarray}
%\begin{eqnarray}
%\ket{r_{p}} = &&\sum_{\substack{n_{p+1},..n_{k-1},n_k\\r_{p+1},...,r_{k-2},r_{k-1}}} R^{n_{p}}_{r_{p},r_{p+1}}\cdot\cdot\cdot R^{n_{k-1}}_{r_{k-2},r_{k-1}}R^{n_k}_{r_k} \nonumber\\
%&&\times \ket{n_{p+1},...,n_{k-1},n_k}
%\end{eqnarray}
%
%With the renormalized basis definied above, wavefunction at site $p$ is
%
%\begin{equation}
%\ket{\Psi} = \sum_{n_p,l_{p-1},r_p} C^{n_p}_{l_{p-1},r_p} \ket{l_{p-1},n_p,r_p}
%\end{equation}
