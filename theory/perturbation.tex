\subsection{$N$-electron valence state perturbation theory} 

We now briefly review the strongly contracted NEVPT2\cite{angeli_n-electron_2001, angeli_n-electron_2002}, a second-order 
multireference perturbation theory. The target zeroth-order wave function is defined as 
\begin{equation}
P_{CAS}HP_{CAS} \ket{\Psi^{(0)}} = E^{(0)} \ket{\Psi^{(0)}}
\end{equation}
where $P_{CAS}$ is the projector onto the CAS space. 
The zeroth order Hamiltonian is chosen in the form given by Dyall\cite{dyall_choice_1995}:
\begin{equation}
  H^D = H_i + H_v + C
\end{equation}
where $H_i$ is a one-electron (diagonal) operator in the non-active (core and external) subspace:
\begin{equation}
  H_i = \sum_{i,\sigma}^{core} \epsilon_i  a^\dagger_{i,\sigma} a_{i,\sigma} + \sum_{r,\sigma}^{virt} \epsilon_r  a^\dagger_{r,\sigma} a_{r,\sigma}
\end{equation}
and $\epsilon_i $ and $\epsilon_r$ are orbital energies; 
$H_v$ is a two-electron operator confined to the active space:
\begin{equation}
  H_v = \sum_{ab,\sigma}^{act} h^{eff}_{ab}  a^\dagger_{a,\sigma}  a_{b,\sigma} + \sum_{abcd,\sigma,\eta}^{act} \braket{ab|cd}  a^\dagger_{a,\sigma} a^\dagger_{b,\eta} a_{d,\eta} a_{c,\sigma}
\end{equation}
where $h_{ab}^{eff} = h_{ab} + \sum_{i}^{core} (2\braket{ai|bi}-\braket{ai|ib})$;
and $C$ is a constant to ensure that $H^D$ is equivalent to the full Hamiltonian within the CAS space.

The zeroth order wave functions external to the CAS space are referred to as the ``perturber functions''. A perturber function is written  as $\ket{\Psi^{(k)}_{l,\mu}}$.
It belongs to a CAS space is denoted $P_{S_l^{(k)}}$, where $k$ is the number of electrons promoted to ($k>0$) or removed from ($k<0$) the active space, $l$ denotes the occupancy of nonactive orbitals and $\mu$ enumerates the various perturber functions in $S_l^{(k)}$.  
In strongly contracted NEVPT2, only one perturber function for each $S_l^{(k)}$ subspace is used. The perturber function is defined as $\ket{\Psi^{(k)}_l} = S^{(k)}_l H \ket{\Psi^{(0)}}$, where $P_{S_l^{(k)}}$ is the projector onto the space. The energy of
the perturber function is
\begin{equation}
  E^{(k)}_l = \frac{\bra{\Psi^{(k)}_l}H^D\ket{\Psi^{(k)}_l}}{\braket{\Psi^{(k)}_l|\Psi^{(k)}_l}}
\end{equation}
The zeroth order Hamiltonian then becomes
\begin{equation}
  H_0 = \sum_{kl} \ket{\Psi^{(k)'}_l}E^{(k)}_l \bra{\Psi^{(k)'}_l} +\ket{\Psi^{(0)}}E^{(0)} \bra{\Psi^{(0)}}
\end{equation}
where $\ket{\Psi^{(k)'}_l}$ corresponds to the normalized $\ket{\Psi^{(k)}_l}$.

The bottleneck in SC-NEVPT2 is the evaluation of the energies of the perturber functions, 
where up to the 4RDM appears. This 4RDM is subsequently contracted with 
two electron integrals in the active space to form auxiliary matrices.\cite{ angeli_n-electron_2002} 
Computing the 4RDM requires $O(k^8M^2+k^5M^3)$ cost, much higher than that of a standard DMRG energy optimization.
To minimize the cost, it is generally necessary to evaluate the 4RDM using a lower bond dimension than is used in the DMRG energy optimization of $\ket{\Psi^{(0)}}$.
To obtain an MPS approximation of lower bond dimension for the reference, 
we  ``compress'' the converged reference MPS of a larger bond dimension.
 This is done in practice by carrying out a ``reverse schedule'' set of sweeps
in a DMRG code, where $M$ is decreased~\cite{olivares-amaya_ab-initio_2015}. All of our DMRG-SC-NEVPT2 calculations use reference MPS that have been 
compressed in this way to compute the RDM's. The compressed bond dimension is
denoted $M'$.

Further, to avoid storage problems in our implementation of DMRG-SC-NEVPT2, 
the 4-RDM is computed on the fly. In principle, it is possible to 
completely avoid generating the 4-RDM by directly computing the auxiliary matrices (i.e. with the integrals already contracted). 
This is the approach taken, for example, in Kurashige and Yanai's original implementation of DMRG-CASPT2
%FIXME
%interface between Block and other software.
~\cite{kurashige_second-order_2011}, where the auxiliary matrices can be computed with a cost scaling similar to that of the 3-RDM,
using special kinds of DMRG renormalized operators as intermediates. However, the Dyall Hamiltonian in NEVPT2 
involves all the active-space two-electron integrals, and this is more complicated to handle than the diagonal Fock energies that appear in the CASPT2 zeroth 
Hamiltonian (in its canonical representation). Thus, \textcolor{red}{although the scaling of such an algorithm may be lower}, it is complicated to implement
and the computational prefactor is large due to the many different intermediates that need to be considered. We have implemented
such an algorithm only in our reference NEVPT2 code that uses CASCI type reference wavefunctions, and not in our DMRG-SC-NEVPT2 code.

%% We also note that  Therefore, the bond dimension of MPS in 4RDM calculations is limited. With the same bond dimension, MPS from a ``reverse schedule'' sweep is usually better optimized that MPS from standard sweep schedule. All of our NEVPT calculations use MPS optmized through ``reverse schedule''.




%The zero-order wave function different from the $\ket{\Psi^{(0)}}$ are referred as the ``perturber functions'', and belong to CAS-CI spaces with different occupation patterns of the non-active (core + virtual) orbitals. The wave function and the space are denoted as $\ket{\Psi^{k}_l}$ and $S_l^{k}$, with $k$ stands for the number of electron promoted to ( or from) active space ( $-2\le k \le 2$), and $l$ describes the occupation pattern in inactive orbitals. 
%
%\begin{equation}
%S^k_l = \ket{\Phi_l^{-k}} \{ \ket{\Psi^k_I} \}
%\end{equation}
%
%where $\Phi_l^{-k}$ is non-active part and $\{ \ket{\Psi^k_I} \} $ is formed by valence determinant with $n_v +k $ electron in active space. 
%
%Perturber functions can be solved by digonalization of $ P_{S_l^k} H P_{S_l^k}$. Or
%the coefficient $\ket{\Phi_l^{-k}\Psi_I^k}$ in the first-order correction to the wave function can be obtained by solving the system of linear equations. 
%
%Above approach is called ``totally uncontracted'', because the full dimensionality of $S_l^k$ is exploited. It is very expensive. 
%
%If only the first-order interaction subspace is used for each $S_l^k$. It is called ``partially contracted''.
%In a more drastic simplification, ``strongly contracted'', a one-dimensional contracted subspace of $S_l^k$. 
%\begin{equation}
%  \ket{\Psi_l^k} = P_{S_l^k} H \ket{\Psi_m^{(0)}} \equiv V_l^k \ket{\Psi_m^{(0)}}
%\end{equation}
%
%is defined.
%
%The relation between different approaches was analyzed by Angeli.  
%
%In this paper, analysis and calculations are based for ``strongly contracted'' NEVPT (SC-NEVPT). 
%%In this paper, analysis and calculations are based for "strongly contracted" NEVPT, even though the bottle neck of "strongly contracted" and "partially contracted" NEVPT are the same. 
%
%In SC-NEVPT, bottle neck is to calculate ``energy'' of perturber functions:
%\begin{equation}
%  E_l^{k(0)} = \frac{\bra{\Psi_l^k}H_0\ket{\Psi_l^k}}{\braket{\Psi_l^k|\Psi_l^k}}  = \frac{\bra{\Psi_l^k}H_0\ket{\Psi_l^k}}{N_l^k}
%\end{equation}
%
%With Dyall's approximation to the electronic Hamiltonian, the active part and the inactive part of the wavefunction are indepedent.
%
%\begin{equation}
%  \begin{split}
%  {N_l^k}E_l^{k(0)} &= \bra{\Psi_l^k}H^D_0\ket{\Psi_l^k} \\
%  &= \bra{\Psi_l^k}H^D_i\ket{\Psi_l^k} + \bra{\Psi_l^k}H^D_v\ket{\Psi_l^k}  \\
%  &= \Delta \epsilon_l^k + \bra{\Psi_m^{(0)}}V_l^{k\dagger}H^D_vV_l^k\ket{\Psi_m^{(0)}} \\
%  &= \Delta \epsilon_l^k + N_l^kE^{(0)}_m + \bra{\Psi_m^{(0)}}V_l^{k\dagger}[H^D_v, V_l^k]\ket{\Psi_m^{(0)}}
%  \end{split}
%\end{equation}
%
%$\Delta \epsilon_l^k$ comes from the orbital energies of electrons in virtual space and holes in core space. $H^D_v$ is the effective two-electron Hamiltionian in active space. $\bra{\Psi_m^{(0)}}V_l^{k\dagger}[H^D_v, V_l^k]\ket{\Psi_m^{(0)}}$ can be calculated from the contraction between the ``excitation integral'' and auxiliary matrices (a citation), which is the contraction of RDM (up the fourth order) and one-electron and two-electron Hamiltonian in active space. The 4-RDM could be calculated on the fly to avoid storaging them.
%
%Like Kurashige and Yanai's implementation of DMRG-CASPT2, these auxiliary matrices can be computed with a scaling same with that of 3-RDM by make special types of operators, which are the contraction of integral and creation (and annihilation) operators. However, the two-electron Hamiltonian is much more complex that the Fork operators (which is diagonal if canonical orbitals used) in CASPT2. Many types of contracted operators would be needed in DMRG frame. It is the very hard to implement and the prefactor is big even though the scaling is better. Therefore, We only contracted integral and operators in our NEVPT2 for ci type reference functions not for DMRG wave function.

%, such as
%
%\begin{equation}
%\bra{\Psi_m^{(0)}}(a^\dagger_i a_ja_k)^\dagger [H_0,a^\dagger_l a_m a_n]\ket{\Psi_m^{(0)}}
%\end{equation}
% with all subscript indexes are orbital in active space. 
% 
