\section*{Apppendix A: Edge cases for number patterns in RDM}
All number patterns in N-RDM, \{$n_\mathcal{S}$,$n_\mathcal{D}$,$n_\mathcal{E}$\}, need to be computed if 
\begin{align}
0\le n_\mathcal{S} \le N \\
0\le n_\mathcal{E} \le N-1 \\
1\le n_\mathcal{D} \le 4 \\
n_\mathcal{S} +n_\mathcal{D} +n_\mathcal{E} =2N \\
\end{align}

However, for some RDM elements, whose indices cannot be aranged as what we discribed above, because too many their indices lie in first or last site. We need additional number patterns to compute them.

For the first step of DMRG sweep, additional patterns are added. These patterns satisfies
\begin{align}
N+1\le n_\mathcal{S} \le min(4,2N) \\
0\le n_\mathcal{D} \le 4 \\
n_\mathcal{S} +n_\mathcal{D} +n_\mathcal{E} =2N \\
\end{align}

For the last step of DMRG sweep, additional patterns are also added. These patterns satisfies
\begin{align}
N\le n_\mathcal{E} \le min(4,2N) \\
0\le n_\mathcal{D} \le 4 \\
n_\mathcal{S} +n_\mathcal{D} +n_\mathcal{E} =2N \\
\end{align}

Because $\mathcal{S}$ (or $\mathcal{E}$) is composed of single site, expectation value of an operator with more than two $\hat{a}$ or two $\hat{a}^\dagger$ on it is zero. The number of indices on single sites should not be more than 4.

\section*{Apppendix B: Restriction on type patterns in RDM}
It is a waste of resource to compute RDM element which is zero or redundant. And the RDM element can be computed through multiple types of combinations of operators. We implemented some restrictions to skip calculations for zero and redundant RDM elements.

The operators $\hat{a}_i$ and $\hat{a}^\dagger_j$ are permutated to make their orbitals indices not decrease. Only the permutation between two operators with different orbital indices are needed.
If $\hat{a}_i$ is ahead of $\hat{a}^\dagger_j$ and $i=j$, the compound operators are not needed in RDM calcualtion. For a single site block, the type with $\hat{a}$ ahead of $\hat{a}^\dagger$ is omitted. For other blocks, the loop of for $\hat{o}_i\hat{o}_j\hat{o}_k\dots$ should be $i \le j \le k\dots$ ( '$=$' is only valid when the two nearby operators in normal order). 


And the RDM ( not transition RDM) is symmetry matrix. Use 3-RDM as example 
%\begin{equation}
%  \bra{\Psi}\hat{a}_i^\dagger\hat{a}_j^\dagger\hat{a}_k\hat{a}_l\ket{\Psi} = \bra{\Psi}\hat{a}_l^\dagger\hat{a}_k^\dagger\hat{a}_j\hat{a}_i\ket{\Psi}^* = \bra{\Psi}\hat{a}_l^\dagger\hat{a}_k^\dagger\hat{a}_j\hat{a}_i\ket{\Psi} = \bra{\Psi}(\hat{a}_i^\dagger\hat{a}_j^\dagger\hat{a}_k\hat{a}_l)^\dagger\ket{\Psi}
%\end{equation}
\begin{equation}
  \langle \hat{a}_i^\dagger\hat{a}_j^\dagger\hat{a}^\dagger_k\hat{a}_l\hat{a}_m\hat{a}_n\rangle = \langle(\hat{a}_i^\dagger\hat{a}_j^\dagger\hat{a}^\dagger_k\hat{a}_l\hat{a}_m\hat{a}_n)^\dagger\rangle
  = \langle \hat{a}_n^\dagger\hat{a}_m^\dagger\hat{a}_l^\dagger\hat{a}_k\hat{a}_j\hat{a}_i\rangle
\end{equation}
We can always use transpose and permutation to make operators begin with $\hat{a}^\dagger$ rather than $\hat{a}$ (the beginning is the operator with smallest orbital indice). Therefore, all types pattern begin with $\hat{a}$ is omitted. 

Similarly, if operators begin with $\hat{a}^\dagger_i\hat{a}_i$, $\hat{a}^\dagger_i\hat{a}_i\hat{a}^\dagger_j\hat{a}_j$ or $\hat{a}^\dagger_i\hat{a}_i\hat{a}^\dagger_j\hat{a}_j\hat{a}^\dagger_k\hat{a}_k$, the remaining part of operators also has transpose and permutation symmetry.

\begin{equation}
\begin{aligned}
  &\langle\hat{a}^\dagger_i\hat{a}_i(\hat{o}_j \hat{o}_k\dots)\rangle \\
  = &\langle(\hat{a}^\dagger_i\hat{a}_i (\hat{o}_j \hat{o}_k\dots))^\dagger\rangle \\
  = &\langle(\hat{o}_j \hat{o}_k\dots)^\dagger(\hat{a}^\dagger_i\hat{a}_i)^\dagger\rangle \\
  = &\langle\hat{a}^\dagger_i\hat{a}_i(\hat{o}_j \hat{o}_k\dots)^\dagger\rangle
\end{aligned}
\end{equation}
where $i < j\le k\dots$

Therefore, there is no need to compute the types operators in which $\hat{a}$ follows $\hat{a}^\dagger_i\hat{a}_i$.

%Combine Permutation and transpose 
%Redundant terms
