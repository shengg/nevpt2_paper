\subsection{Poly(p-phenylene vinylene)}

Poly(p-phenylene vinylene) is a well-known light emitting conjugated polymer. It has several low-lying states that participate in its photophysics 
and has been extensively investigated with many theoretical methods, both at the semi-empirical~\cite{shukla_correlated_2002,bursill_symmetry-adapted_2009,beljonne_theoretical_1995} and ab-initio level~\cite{xxx} \textcolor{red}{I'm sure there are ab-initio calculations}. 
%% However, we ab initio multireference correlated wave function calculations have not been reported for longer oligomers
%% due to the large number of active orbitals. \textcolor{red}{
%% were not often reported, because of the difficulty to include static and dynamic correlation together for a complex system.
We use PPV-3 as an example to demonstrate the quality of DMRG-SC-NEVPT2 for low-lying excited states in a prototypical conjugated molecule.

We used a cc-pVDZ basis and a (22e,22o) active space containing the 22 conjugated $\pi, \pi^*$ orbitals. We obtained the ground state equilibrium geometry of PPV-3 via DMRG-CI geometry optimization\cite{hu_excited-state_2015} as implemented with an interface between Block and ORCA\cite{neese_orca_2012}, starting from a guess from the DFT/B3LYP optimized geometry. The geometry optimization was first run at $M$=1000, and near convergence with $M$=2000. Canonical Hartree-Fock orbitals were used 
in the DMRG calculations in the geometry optimization.
%For the first optically bright state, we also started from the DFT guess ground state geometry.

We then performed state-averaged DMRG-CI calculations using both the canonical orbitals of the ground state, as well as split-localized orbitals. For the canonical orbital DMRG-CI calculations, we performed
state-averaged calculations within the $^1A_g$ (2 states) and $B_u$ (3 states) symmetry sectors.
In the case of the split-localized orbital DMRG calculations, we carried out state-averaged calculations over 6 states.
To identify the character of the excited states in the split-localized basis, we computed the transition 1-RDM and transition 2-RDM from the ground state, and 
identified the state according to  its excitation signature.

Based on the chosen active space, the (canonical) DMRG-CI calculation 
identified $3^1B_u$  as the first optically bright state, with a HOMO$\rightarrow$ LUMO excitation signature.  $2^1A_g$ in a state-average calculation 
with 2 states (\textcolor{red}{this can't be right - the lower Ag state is still dark} $3^1A_g$ when more states were averaged) is the first dark state, with a HOMO-1$\rightarrow$ LUMO + HOMO$\rightarrow$ LUMO+1 and HOMO,HOMO$\rightarrow$ LUMO,LUMO excitation signature.
Due to the lack of dynamic correlation ($\sigma-\pi$ interaction), the energy ordering of the DMRG-CI excited states is significantly different 
from experimental~\cite{shukla_correlated_2002,bursill_symmetry-adapted_2009}.

After including dynamic correlation through DMRG-NEVPT2, the energy first bright state was much lowered and it became the first excited state ($1^1B_u$). 
This is similar to low-lying excited states of polyenes, where dynamic correlations lower the ionic excited state relative to covalent excited state.
However, the energy order did not converge with the bond dimension in the calculation (Table~\ref{table:canonical}).


\begin{table}
   \caption{Excitation energy (eV) calculated from DMRG-NEVPT2 with canonical orbitals. States are labeled according to energies with M= 750. }
  \label{table:canonical}
\begin{tabular}{ccccc}
\hline
  state & M=250 & M=500 & M=750 & DMRG-CI(M=750)\\
\hline
  $2^1A_g$ & 4.742 &  4.669  &   4.657   &   4.991 \\
  $1^1B_u$ & 4.027 &  4.191  &   4.180   &   5.269 \\
  $2^1B_u$ & 4.449 &  4.486  &   4.264   &   4.740 \\
  $3^1B_u$ & 5.090 &  4.328  &   4.600   &   5.132 \\
\hline
\end{tabular}
\end{table}

Localized orbitals usually give a better convergence for DMRG calculation.\cite{olivares-amaya_ab-initio_2015} Therefore, further DMRG-NEVPT2 calculations were carried out based on split-localized orbitals (localize the occupied and unoccupied $\pi$ orbitals separately with Pipek-Mezey method\cite{pipek_fast_1989}). 
The NEVPT2 energies were calculated for lowest 6 states (based on DMRG-CI energy). The excitation energies were converged into less than $0.6 meV$with bond dimension (Table~\ref{table:local}). They were accurate enough to determine the energy order of low-lying excite states. 
The difference between calculated excitation energies from canonical orbital and those from local orbitals was big, and it might not be the same even when bond dimension is infinite. One reason is that the molecular geometry was optimized without symmetry restriction. Some integral terms, which should be zero according symmetry, were not exact zero. 
%When local orbital orbitals were used, the symmetry restriction on the integral was removed. 

\begin{table}
   \caption{Excitation energy (eV) calculated from DMRG-NEVPT2 with localized orbitals. States are labeled according to excitation signatures and energies with M= 750. }
  \label{table:local}
\begin{tabular}{ccccc}
\hline
  state &  M=500 & M=750 & DMRG-CI(M=750)\\
\hline
  $1^1B_u$ & 3.8628   &   3.8625   & 5.1312    \\
  $2^1B_u$ & 4.4980   &   4.4978   & 4.4953    \\
  $2^1A_g$ & 4.6634   &   4.6634   & 4.8456    \\
  $3^1B_u$ & 4.8268   &   4.8262   & 4.7542    \\
  $4^1B_u$ & 4.8441   &   4.8443   & 4.7572    \\
\hline
\end{tabular}
\end{table}

The excitation energy of first bright state (3.86eV) agrees well with results from other theoretical studies (3.88eV \cite{beljonne_theoretical_1995}, 3.91eV\cite{shukla_correlated_2002}). 
However, it deviated from the experimental value (3.5eV\cite{woo_optical_1993},3.43eV\cite{gelinck_measuring_2000}) a lot. The reasons were that PPV oligomer was substituted with tertiary butyl end-group to become soluble in the experiment and that the solvent was not considered in the calculations.
