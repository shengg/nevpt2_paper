\subsection{Poly(p-phenylene vinylene)}

Poly(p-phenylene vinylene) is a well-known light emitting conjugated polymer. It has several low-lying states that participate in its photophysics 
and has been extensively investigated with many theoretical methods, for example various semi-empirical methods~\cite{beljonne_theoretical_1995}, DMRG based on the Pariser-Parr-Pople (PPP) model Hamiltonian~\cite{lavrentiev_theoretical_1999, shukla_correlated_2002,bursill_symmetry-adapted_2009}, time-dependent density functional theory~\cite{han_time-dependent_2004} and symmetry-adapted cluster configuration interaction (SAC-CI)~\cite{saha_investigation_2007}. 
Here we examine the excited states and their energies using DMRG-SC-NEVPT2.
We limit our calculations to PPV3 (three phenyl rings) to demonstrate the capabilities of DMRG-SC-NEVPT2 for low-lying excited states in conjugated molecules.

The cc-pVDZ basis set~\cite{dunning1989gaussian} and a (22$e$, 22$o$) active space containing the 22 conjugated $\pi, \pi^*$ orbitals were used. The ground state equilibrium geometry of PPV3 was obtained via DMRG-CI geometry optimization~\cite{hu_excited-state_2015}, which is implemented within an interface between \textsc{Block} and \textsc{ORCA}\cite{neese_orca_2012}, starting from the DFT/B3LYP optimized geometry. The DMRG-CI geometry optimization was first run at $M$ = 1000, and near convergence with $M$ = 2000. Canonical Hartree-Fock orbitals were used 
in the DMRG calculations in the geometry optimization.
%For the first optically bright state, we also started from the DFT guess ground state geometry.

Compared to canonical orbitals, localized orbitals usually give better convergence in DMRG calculations.~\cite{olivares-amaya_ab-initio_2015} Therefore, further state-averaged DMRG-CI calculations were carried out for the 6 lowest singlet states, using split-localized orbitals (where the occupied and unoccupied $\pi$, $\pi*$ orbitals are separately localized with the Pipek-Mezey method~\cite{pipek_fast_1989}). The energies were converged to $0.2mEh$ with a bond dimension of $M$ = 3200. The excited states were identified according to their excitation signature from the ground state through the transition 1-RDM and 2-RDM
in the canonical Hartree-Fock basis. The symmetry of the states was also determined by the excitation signature: as the ground-state
 is an $A_g$ state, excitations between two orbitals of different symmetries lead to a $B_u$ excited state, and excitations between two orbitals of the same symmetry lead to an $A_g$ state.

Due to the lack of $\pi-\sigma$ interaction and dynamical correlation, the energy ordering of excitations in DMRG-CI is qualitatively different
from experiment, and the state with the HOMO$\rightarrow$LUMO excitation signature was not the lowest one in energy.
To obtain the correct state ordering, dynamic correlation was further included using DMRG-SC-NEVPT2. The DMRG-CI wavefunctions were compressed
to a smaller bond dimension of $M'$=500 and 750. The difference in excitation energy from these two different $M'$ bond dimensions
in the DMRG-SC-NEVPT2 calculation was less than $0.6 meV$ (Table~\ref{table:local}), and thus the error from incomplete $M'$ can
effectively be ignored.
The excitation energy of the first bright state as computed by DMRG-SC-NEVPT2 ($3.86 eV$) now agrees well with
the experimental number from fluorescence in vacuo by extrapolation the linear fits to $n_{solv}=1$ ($3.69 eV$)~\cite{gierschner_fluorescence_2002} (Table~\ref{table:compare}) and the first bright ($B_u$ state) is found to lie below the $2A_g$ state, as required for efficient light emission. We also find a second $B_u$ state slightly
below the $2A_g$ state in energy. The electronic signatures of the states are shown in Table~\ref{table:local}.

\begin{table}
\caption{Vertical excitation energies (eV) of PPV3 in vacuum from different methods.}
\label{table:compare}
\begin{tabular}{ccccc}
  \hline
  \hline
State  & DMRG-SC-NEVPT2 & DMRG/PPP & SAC-CI & expt\\
\hline
$1^1B_u$ & 3.86   &3.52\footnote{\label{fn:dmrg_1999}Ref.~\onlinecite{lavrentiev_theoretical_1999}}, 3.91\footnote{Ref.~\onlinecite{shukla_correlated_2002}}, 3.751\footnote{\label{fn:dmrg_2009}Ref.~\onlinecite{bursill_symmetry-adapted_2009}}, & 3.57\footnote{Ref.~\onlinecite{saha_investigation_2007}} & 3.69\footnote{Ref.~\onlinecite{gierschner_fluorescence_2002}} \\
%$1^1B_u$ & 3.863   &3.52\footnote{\label{fn:dmrg_1999}Ref.~\onlinecite{lavrentiev_theoretical_1999}}, 3.751\footnote{\label{fn:dmrg_2009}  Ref.~\onlinecite{bursill_symmetry-adapted_2009}}, 3.88\footnote{Ref.~\onlinecite{beljonne_theoretical_1995}},3.91\footnote{Ref.~\onlinecite{shukla_correlated_2002}}, 3.68\footnote{Ref.~\onlinecite{gierschner_fluorescence_2002}} & 3.69\footnote{Ref.~\onlinecite{gierschner_fluorescence_2002}} \\
$2^1A_g$ & 4.66   &4.06\footref{fn:dmrg_1999}, 4.618\footref{fn:dmrg_2009}& & \\
\hline
\hline
\end{tabular}
\end{table}




\begin{table*}
  \centering
  \caption{Excitation energy (eV) calculated from DMRG-SC-NEVPT2 ($M'$=500, 750) with localized orbitals.
    Excitations energies of DMRG-CI ($M$=3200) also shown for comparison. The transition signatures are calculated from the
    DMRG-CI wave function with $M$ = 3200. The excitation coefficient of the transition $i\rightarrow j$ ($i,j \rightarrow k,l$) is given by $\frac{1}{\sqrt{2}}\bra{\Psi}a^\dagger_j a_i\ket{1^1A_g}$ ( $\frac{1}{2}\bra{\Psi}a^\dagger_l a^\dagger_k a_j a_i\ket{1^1A_g}$).
The transition labels $n\rightarrow m'$ are as follows: 1, 2, 3 \ldots denote HOMO, HOMO-1, HOMO-2 \ldots canonical orbitals, while $1'$, $2'$, $3'$ \ldots denote LUMO, LUMO+1,LUMO+2 canonical orbitals. }
  \label{table:local}
  \begin{tabular}{ccccc}
    \hline
\hline
symmetry & excitation signature &  $M'$=500 & $M'$=750 & DMRG-CI($M$=3200) \\

\hline
  $1^1B_u$ & 0.83 ($1\rightarrow1')$ & 3.86   &   3.86   & 5.13 \\
  $2^1B_u$ & 0.51 ($5\rightarrow 1'$) + 0.47 (1$\rightarrow 3'$) & 4.50   &   4.50   & 4.50    \\
  $2^1A_g$ & 0.35 ($2\rightarrow 1'$) + 0.30 (1$\rightarrow 2'$) & 4.66   &   4.66   & 4.85    \\
  &   + 0.47 ($1,1\rightarrow1',1'$)  & & &\\
  $3^1B_u$ & many singly excited terms & 4.83   &   4.83   & 4.75    \\
  $4^1B_u$ & many singly excited terms & 4.84   &   4.84   & 4.76    \\
\hline
\hline
\end{tabular}
\end{table*}
