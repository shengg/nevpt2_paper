
\section{Introduction}

The density matrix renormalization group (DMRG) \cite{white_density_1992,white_density-matrix_1993} has made it possible to employ large active spaces 
for electronic structure problems in quantum chemistry. It has become straightforward to use DMRG as a robust numerical solver for 
static electron correlation problems. Examples of recent applications include benchmark solutions of small molecules\cite{chan_highly_2002}, multi-center transition metal clusters\cite{sharma_low-energy_2014, olivares-amaya_ab-initio_2015}, as well as molecular crystals\cite{yang_ab_2014}. 

In chemical systems, electron interactions across a wide range of energy scales are important for many problems. Such interactions are, for example, 
important in determining the ordering of the low-lying excited states 
in conjugated molecules, or the correct spin-state splittings in transition metal complexes. Qualitative and quantitative characterizations of 
these systems generally require correlating many electrons within a large number of orbitals. Direct treatment of such problems is generally 
impossible, but indirect methods can be used, for example by separating the treatment of electron correlation inside and outside of the active space. 
So far, there are have been several attempts to apply this strategy with the large active spaces accessible to DMRG, for example
through perturbation theory or configuration interaction on top of a DMRG active space
wavefunction~\cite{kurashige_second-order_2011, sharma_communication:_2014}, or by first performing a canonical transformation to obtain an effective Hamiltonian~\cite{neuscammanreview_2010}.
%On the other hand, dynamic electron correlation is crucial in chemical dynamic processes, which in general involves electron correlation dynamics within a large range of orbitals. 
%% The DMRG method has been embedded with the second-order perturbation theory
%% complete active space perturbation theory (CASPT), and canonical transformation(CT)\cite{neuscamman_review_2010}, etc., to be extended for dynamic correlation. 
In these methods, high order reduced density matrices of the DMRG active space wavefunction are required. Such density matrices 
are simple objects in a theoretical sense, but are computationally non-trivial to obtain.

In this work, we have developed a general algorithm to efficiently compute high order reduced density matrix (RDM) from DMRG wave functions. 
Using these RDM calculations, we may then implement a variety of methods to treat electron correlation external to the DMRG active space.
An example of this, we have implemented the second order $N$-electron valence state perturbation theory (NEVPT2) 
\cite{angeli_introduction_2001,angeli_n-electron_2001, angeli_n-electron_2002}, an intruder-state-free multireference perturbation theory in it strongly-contracted variant, on top
of the DMRG reference wavefunction, a method we call DMRG-SC-NEVPT2.
To demonstrate the potential of this approach, we apply this method to study the potential energy curve of the chromium dimer, and 
to compute the excitation energies of the quasi-one dimensional conjugated poly(p-phenylene vinylene) (PPV). 
The chromium dimer is a particularly demanding small molecule that has been widely studied with many methods\cite{roos_ground_2003,celani_cipt2_2004,angeli_third-order_2006,muller_large-scale_2009,kurashige_second-order_2011,ruiperez_complete_2011,kurashige_multireference_2014,sharma_multireference_2015}, and thus forms a good test bed to demonstrate the performance of DMRG-SC-NEVPT2. Using an active space with 22 orbitals and basis sets extrapolated to the CBS limit, we show that we can
compute both the spectroscopic constants, as well as the full curve,
with an accuracy that compares quite favourably to earlier calculations.
PPV is of interest as the prototypical light-emitting polymer\cite{burroughes_light-emitting_1990,friend_electroluminescence_1999}. This property relies
on the correct ordering of the first optically bright $1^{1}B_{u}$ state
and  dark $2^{1}A_{g}$ state, which in turn requires an appropriate
description of the correlation~\cite{shukla_correlated_2002, bursill_symmetry-adapted_2009,beljonne_theoretical_1995}.
%% The first  whose polaronic feature is suggested to be 
%% responsible for the charge-transfer behavior, is estimated to lie below the  with a small gap of 0.7eV\cite{martin_linear_1999}. 
%% While many theoretical studies of PPV have used model Hamiltonians, such as the  Pariser\textendash Parr\textendash Pople (PPP) model, \cite{shukla_correlated_2002, bursill_symmetry-adapted_2009} or other semi-empirical  methods \cite{beljonne_theoretical_1995} due to the large system size, 
With DMRG-SC-NEVPT2, we show that we can compute the excitation 
energies and energy ordering of the different states accurately starting 
from the full-valence active $\pi$ space of the molecule.
%need to say something more...

%In this work, we introduce an algorithm which generally applies to any order of (transition) particle density matrices for DMRG wave functions. We first give a brief review on the DMRG method in Sec. Further we present a general algorithm to evaluate any order particle density matrices in Sec.. We show the application of this algorithm in the DMRG-NEVPT2 method, to study the static and dynamic correlation effects of photovoltaic polymer PPV in Sec.
