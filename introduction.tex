
\section{Introduction}

In manly non-trivial quantum chemistry applications, such as excited states, non-equilibrium geometries and transition-metal compounds, the methods based on mean-field electron interaction cannot even give qualitative descriptions. 
The deviation from mean-field electron interation, called electron correlation, is usually split into two parts: static or non-dynamic correlation and dynamic correlation.
Static correlation is among the near-degenerate orbitals and can be effectively treated by allowing all possible distribution of electrons in these near-degenerate orbtals (active orbitals).  
One of the models based on active orbitals is complete active space self-consistent field (CASSCF) method. Dynamic correlation usually does not make as strong influences on electronic structure as the static correlation does. Dynamic correlation can be effectively discribed by low-level theories, such as perturbation theories (PT), configuration interaction singles (CISD)  and doubles or coupled cluster theory (CC). If dynamic correlation and static correlation are both needed, such as the determination of energy order of excited states in conjugated molecules or transition metal compounds, the multi-reference (MR) versions of the theories mentioned above are needed. They are called multi-reference perturbation theories (MRPT)\cite{andersson_second-order_1990}, multi-reference configuration interaction (MRCI)\cite{buenker_individualized_1974} or multi-reference coupled clusters theories (MRCC) \cite{oliphant_multireference_1991}.

In CASSCF and MR methods (MRPT, MRCI and MRCC) based on CASSCF wave functions, static correlation within the active orbitals is solved exactly through the full configuration interaction (FCI). The exponential increasing of computations of FCI limits the number of active orbitals. The density matrix renormalization group (DMRG) \cite{white_density_1992,white_density-matrix_1993, ostlund_thermodynamic_1995, rommer_class_1997, schollwock_density-matrix_2005, schollwock_density-matrix_2011} has made it possible to employ large active spaces 
for electronic structure problems in quantum chemistry.\cite{white_ab_1999, chan_highly_2002, legeza_controlling_2003,moritz_decomposition_2007, verstraete_matrix_2008, marti_density_2008, zgid_spin_2008, marti_density_2010, kurashige_high-performance_2009, luo_optimizing_2010, chan_density_2011, marti_new_2011, sharma_spin-adapted_2012, wouters_longitudinal_2012, wouters_thouless_2013, kurashige_entangled_2013, wouters_communication:_2014} It has become straightforward to use DMRG as a robust numerical solver for 
static electron correlation problems. Examples of applications include benchmark solutions of small molecules\cite{chan_highly_2002,legeza_controlling_2003,luo_optimizing_2010, olivares-amaya_ab-initio_2015}, multi-center transition metal clusters\cite{kurashige_high-performance_2009,marti_new_2011, kurashige_entangled_2013, sharma_low-energy_2014}, as well as molecular crystals\cite{yang_ab_2014}. It is a good alternative of FCI.

There were also several attempts to add dynamic correlation on the top of DMRG ansatz.
For example, DMRG-CT~\cite{neuscamman_review_2010}, DMRG-CASPT2 ~\cite{kurashige_second-order_2011}, DMRG-cu(4)-MRCI ~\cite{saitow_fully_2015} in the internal-contraction(IC) framework and MPS-PT~\cite{sharma_communication:_2014}, MPS-LCC ~\cite{sharma_multireference_2015} in fully uncontracted framework.
%On the other hand, dynamic electron correlation is crucial in chemical dynamic processes, which in general involves electron correlation dynamics within a large range of orbitals. 
%% The DMRG method has been embedded with the second-order perturbation theory
%% complete active space perturbation theory (CASPT), and canonical transformation(CT)\cite{neuscamman_review_2010}, etc., to be extended for dynamic correlation. 
In internal-contraction methods, high order reduced density matrices of the DMRG active space wavefunction are required. Such density matrices 
are simple objects in a theoretical sense, but are computationally non-trivial to obtain.

In this work, we have developed a general algorithm to efficiently compute high order reduced density matrix (RDM) from DMRG wave functions. 
Using these RDM calculations, we may then implement a variety of methods to treat electron correlation external to the DMRG active space.
An example of this, we have implemented the second order $N$-electron valence state perturbation theory (NEVPT2) 
\cite{angeli_introduction_2001,angeli_n-electron_2001, angeli_n-electron_2002}, an intruder-state-free multireference perturbation theory in its strongly-contracted variant, on top
of the DMRG reference wavefunction, a method we call DMRG-SC-NEVPT2.
To demonstrate the potential of this approach, we apply this method to study the potential energy curve of the chromium dimer, and 
to compute the excitation energies of the quasi-one dimensional conjugated poly(p-phenylene vinylene) (PPV). 
The chromium dimer is a particularly demanding small molecule that has been widely studied with many methods\cite{roos_ground_2003,celani_cipt2_2004,angeli_third-order_2006,muller_large-scale_2009,kurashige_second-order_2011,ruiperez_complete_2011,kurashige_multireference_2014,sharma_multireference_2015}, and thus forms a good test bed to demonstrate the performance of DMRG-SC-NEVPT2. Using an active space with 22 orbitals and basis sets extrapolated to the CBS limit, we show that we can
compute both the spectroscopic constants and the full curve,
with an accuracy that compares quite favourably to earlier calculations.
PPV is of interest as the prototypical light-emitting polymer\cite{burroughes_light-emitting_1990,friend_electroluminescence_1999}. This property relies
on the correct ordering of the first optically bright $1^{1}B_{u}$ state
and  dark $2^{1}A_{g}$ state, which requires an appropriate
description of electron correlation~\cite{beljonne_theoretical_1995,lavrentiev_theoretical_1999,shukla_correlated_2002, han_time-dependent_2004, saha_investigation_2007, bursill_symmetry-adapted_2009}.
%% The first  whose polaronic feature is suggested to be 
%% responsible for the charge-transfer behavior, is estimated to lie below the  with a small gap of 0.7eV\cite{martin_linear_1999}. 
%% While many theoretical studies of PPV have used model Hamiltonians, such as the  Pariser\textendash Parr\textendash Pople (PPP) model, \cite{shukla_correlated_2002, bursill_symmetry-adapted_2009} or other semi-empirical  methods \cite{beljonne_theoretical_1995} due to the large system size, 
With DMRG-SC-NEVPT2, we show that we can compute the excitation 
energies and energy ordering of the different states accurately starting 
from the full-valence active $\pi$ space of the molecule.

