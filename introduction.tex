
\section{introduction}

The density matrix renormalization group (DMRG) \cite{white_density_1992,white_density-matrix_1993} has made it accessible to employ large active spaces 
for electronic structure problems in quantum chemistry. Nowadays, it has become quite straightforward to use DMRG as a robust numerical solver for 
static electron correlation problems. Examples are benchmark solutions of small molecules\cite{chan_highly_2002}, transition metal 
clusters\cite{sharma_low-energy_2014, olivares-amaya_ab-initio_2015}, also molecular crystals\cite{yang_ab_2014}. 

In chemical systems, electron interaction across large energy scales brings many other important chemistry, such as low-lying excited states in 
conjugated polymers, high-spin ground state in transition metals complexes. Qualitative and quantitative characterizations of 
these systems generally require correlating many electrons within a large number of orbitals. Direct treatment of such problem is generally 
impossible, but indirect methodology can apply, for example, by expanding the electron correlation outside active spaces. So far, there are 
methods which follow with this strategy with using large active spaces, for example, to apply perturbation theory to a large active space, or 
perform canonical transformation to get an effective Hamiltonian.
%On the other hand, dynamic electron correlation is crucial in chemical dynamic processes, which in general involves electron correlation dynamics within a large range of orbitals. 
The DMRG method has been embedded with the second-order perturbation theory\cite{kurashige_second-order_2011, sharma_communication:_2014}, 
complete active space perturbation theory (CASPT), and canonical transformation(CT)\cite{neuscamman_review_2010}, etc., to be extended for dynamic correlation. In these methods, high order reduced density matrices are generally needed, which are theoretically straightforward yet computationally non-trivial.

In this work, we developed a general way to compute high order reduced density matrix (RDM) for DRMG wave functions. 
On the top of RDM calculations, we implemented the second order $n$-electron valence state perturbation theory (NEVPT2)
\cite{angeli_introduction_2001,angeli_n-electron_2001, angeli_n-electron_2002}, an intruder-state-free multireference perturbation theory, with DMRG reference wavefunctions, called DMRG-NEVPT2.
Then we applied it to study potential energy curve of chromium dimer and excitation energies of the quasi-one dimensional  photovoltaic molecule poly(p-phenylene vinylene) (PPV). 
The calculations for the potential energy curve of chromium dimer is one of the most notorious and demanding problems in ab-inito quantum chemistry. This system has been widely studied by many methods\cite{roos_ground_2003,celani_cipt2_2004,angeli_third-order_2006,muller_large-scale_2009,kurashige_second-order_2011,ruiperez_complete_2011,kurashige_multireference_2014,sharma_multireference_2015}. It is a good system to demonstrate the capacity of DMRG-NEVPT2. 
PPV has always been of great interest by photophysists and photochemists, as it is well known for its charge-transfer properties upon photoexcitations\cite{burroughes_light-emitting_1990,friend_electroluminescence_1999}. The first optically bright state $1^{1}B_{u}$ whose polaronic feature is suggested to be 
responsible for the charge-transfer behavior, is estimated to lie below the $2^{1}A_{g}$ with a small gap of 0.7eV\cite{martin_linear_1999}. 
Many theoretical studies of PPV are based on model Hamiltonian, such as  Pariser\textendash Parr\textendash Pople (PPP) model, \cite{shukla_correlated_2002, bursill_symmetry-adapted_2009} or semi-empirical quantum chemistry method \cite{beljonne_theoretical_1995}. However, ab initio quantum chemistry calculation were hardly reported due to the unaffordable computations to including the dynamic correlation and static correlation at the same time for such a system with many $\pi$ orbitals, which need to be considered as active space in complete active space (CAS) calculations.
With DMRG-NEVPT2, now it is possible to compute the excitation energies and energy orders of different states accurately.
%need to say something more...

%In this work, we introduce an algorithm which generally applies to any order of (transition) particle density matrices for DMRG wave functions. We first give a brief review on the DMRG method in Sec. Further we present a general algorithm to evaluate any order particle density matrices in Sec.. We show the application of this algorithm in the DMRG-NEVPT2 method, to study the static and dynamic correlation effects of photovoltaic polymer PPV in Sec.
