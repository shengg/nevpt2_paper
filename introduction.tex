
\section{introduction}

In chemical systems, electron interaction across large energy scales brings many other important chemistry, for example, 
low-lying excited states in conjugated polymers, high-spin ground state in transition metal complexes. Quantitative 
characterizations of these systems generally require correlating many electrons within a wide range of orbitals. Directly 
correlating electrons of interest in these problems is general exponentially expensive, while indirect electron correlating 
methodologies has been shown as effective workarounds. So far, strategies of first exactly correlating most chemically centered 
electrons then introducing dynamic correlations to extennal orbital spaces, brings high qualitative description of energy level 
predictions of various metalliorganic complexes (metallic-porphryins), and accurate dissociation curves of small molecules(cite 
takeshi's and eric's paper).

There are several methodologies to introduce the dynamic correlation. One of the methodologies is to recover effective excitations from the 
acturately correlated active space to external space, where excitations can be of singly, doubly or even higher order excitations. 
A representative method of such methodology are canonical transformation (CT) theory (whether we can write Coupled cluster? difference?) which builds 
an effective Hamiltonian based on a reference active space correlated wavefunction. Another kind of methodology is to treat these 
excitations are perturbations to the correlated active space wavefunction. Representative methods are the complete active space 
perturbation theory (CASPT), and N-electron valence perturbation theory (NEVPT). Generally, successful applications of 
these methods rely on 1) an adequate inclusion of static correlations. 2) descriptive dynamic correlation schemes. 
3) acceptable computational cost for both static and dynamic portions. 

Adequately including static correlations can be done by correlating enough number of electrons which are highly possible in centrally 
related to chemical behaviors. Although the scaling of exact static correlation is exponential, the cost can be downgraded to 
polymonial by using methods such as density matrix renormalization group (DMRG). For dynamic correlations, current dynamic correlation methods 
can effectively recover a large portion of correlations with a relatively low cost by only using low order exctations. Specifically, compared to CASPT method, 
NEVPT2 is generally considered as more descriptive because it avoids theoretical disadvantanges such as intruder states 
(other? and whether these disadvantages can be recovered in other methods?).
However, in many problems where the electron entanglement signature is far more complicated and high order excitations are required. 

The major numerical challenge to solve dynamic correlation problem is the high computational scaling of high order reduced density matrices which 
characterizes the excitation behaviors among orbitals. For dynamic correlations based on CASCI reference wavefunction, the computational time and storage 
scales as $O(N^{2a})$ where $N$ is the number of orbitals and $a$ is the order of the excitation. For DMRG method, the general computational time scales 
as $O(N^{2a} M^{2})$, where $M$ is the number of kept states in DMRG calculations. 

In this work, we developed a general way to compute high order reduced density matrix (RDM) for DMRG wave functions. We further utilized 
these RDMs in NEVPT2 method\cite{angeli_introduction_2001, angeli_n-electron_2001, angeli_n-electron_2002}, and revisit the dissociation 
curve of chromium dimer and the exitation energies of poly(p-phenylene vinylene) (PPV). For $Cr_{2}$, the particiation of $4s$ electron in 
the chemical bonding process introduces more chemistry details but is a challenging problem for quantum chemistry. For PPV, the excitation energy 
order is not only important for understanding in its photovotaic behaviors, but also brings more insights about its unique energy structure 
distinct from many other conjugated polymers.

%FURTHER WRITE THE STRUCTURE OF THIS PAPER

%On the other hand, dynamic electron correlation is crucial in chemical dynamic processes, which in general involves electron correlation dynamics within a large range of orbitals. 
%The DMRG method has been embedded with the second-order perturbation theory\cite{kurashige_second-order_2011, sharma_communication:_2014}, 
%complete active space perturbation theory (CASPT), and canonical transformation(CT)\cite{neuscamman_review_2010}, etc., to be extended for dynamic correlation. In these methods, high order reduced density matrices are generally needed, which are theoretically straightforward yet computationally non-trivial to compute.

%The density matrix renormalization group (DMRG) \cite{white_density_1992,white_density-matrix_1993} has made it accessible to employ large active spaces 
%for electronic structure problems in quantum chemistry. Nowadays, it has become quite straightforward to use DMRG as a robust numerical solver for 
%static electron correlation problems. Examples are benchmark solutions of small molecules\cite{chan_highly_2002}, transition metal 
%clusters\cite{sharma_low-energy_2014, olivares-amaya_ab-initio_2015}, also molecular crystals\cite{yang_ab_2014}. 

%In this work, we developed a general way to compute high order reduced density matrix (RDM) for DRMG wave functions. 
%On the top of RDM calculations, we implemented the second order $n$-electron valence state perturbation theory (NEVPT2)
%\cite{angeli_introduction_2001,angeli_n-electron_2001, angeli_n-electron_2002}, an intruder-state-free multireference perturbation theory, with DMRG reference wavefunctions, called DMRG-NEVPT2.
%Then we applied it to study potential energy curve of chromium dimer and excitation energies of the quasi-one dimensional  photovoltaic molecule poly(p-phenylene vinylene) (PPV). 
%The calculations for the potential energy curve of chromium dimer is one of the most notorious and demanding problem in ab-inito quantum chemistry. This system has been widely studied by many methods\cite{roos_ground_2003,celani_cipt2_2004,angeli_third-order_2006,muller_large-scale_2009,kurashige_second-order_2011,ruiperez_complete_2011,kurashige_multireference_2014,sharma_multireference_2015}. It is good system to demonstrate the capacity of DMRG-NEVPT2. 
%PPV has always been of great interest by photophysists and photochemists, as it is well known for its charge-transfer properties upon photoexcitations\cite{burroughes_light-emitting_1990,friend_electroluminescence_1999}. The first optically bright state $1^{1}B_{u}$ whose polaronic feature is suggested to be 
%responsible for the charge-transfer behavior, is estimated to lie below the $2^{1}A_{g}$ with a small gap of 0.7eV\cite{martin_linear_1999}. 
%Many theoretical studies of PPV are based on model Hamiltonian, such as  Pariser\textendash Parr\textendash Pople (PPP) model, \cite{shukla_correlated_2002, bursill_symmetry-adapted_2009} or semi-empirical quantum chemistry method \cite{beljonne_theoretical_1995}. However, ab initio quantum chemistry calculation were hardly reported due to the unaffordable computations to including the dynamic correlation and static correlation at the same time for such a system with many $\pi$ orbitals, which need to be considered as active space in complete active space (CAS) calculations.
%With DMRG-NEVPT2, now it is possible to compute the excitation energies and energy orders of different states accurately.
%need to say something more...

%In this work, we introduce an algorithm which generally applies to any order of (transition) particle density matrices for DMRG wave functions. We first give a brief review on the DMRG method in Sec. Further we present a general algorithm to evaluate any order particle density matrices in Sec.. We show the application of this algorithm in the DMRG-NEVPT2 method, to study the static and dynamic correlation effects of photovoltaic polymer PPV in Sec.
