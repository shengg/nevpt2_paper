
\section{introduction}

The density matrix renormalization group \cite{white_density_1992,white_density-matrix_1993} has made it accessible to employ large active spaces 
for electronic structure problems in quantum chemistry. Nowadays, it has become quite straightforward to use DMRG as a robust numerical solver for 
static electron correlation problems. Examples are benchmark solutions of small molecules\cite{chan_highly_2002}, transition metal 
clusters\cite{sharma_low-energy_2014, olivares-amaya_ab-initio_2015}, also molecular crystals\cite{yang_ab_2014}. 

In chemical systems, electron interaction across large energy scales brings many other important chemistry, such as low-lying excited states in 
conjugated polymers, high-spin ground state in transition metals complexes. Qualitative and quantitative characterizations of 
these systems generally require correlating many electrons within a large number of orbitals. Direct treatment of such problem is generally 
impossible, but indirect methodology can apply, for example, by expanding the electron correlation outside active spaces. So far, there are 
methods which follow with this strategy with using large active spaces, for example, to apply perturbation theory to a large active space, or 
perform canonical transformation to get an effective Hamil
%On the other hand, dynamic electron correlation is crucial in chemical dynamic processes, which in general involves electron correlation dynamics within a large range of orbitals. 
The DMRG method has been embedded with the second-order perturbation theory\cite{kurashige_second-order_2011, sharma_communication:_2014}, 
complete active space perturbation theory (CASPT), and canonical transformation(CT)\cite{neuscamman_review_2010}, etc., to be extended for dynamic correlation. In these methods, high order particle density matrices are generally needed, which are theoretically straightforward yet computationally non-trivial to compute.

In this work, we implemented second order $n$-electron valence state perturbation theory based of DMRG wavefunction (DMRG-NEVPT2) and 
apply it to the quasi-one dimensional  photovoltaic molecule poly(p-phenylene vinylene) (PPV) and chromium dimer. PPV has always been 
of great interest by photophysists and photochemists, as it is well known for its charge-transfer properties upon photoexcitations. Due 
to the highly conjugated structure in PPV, the first optically bright state $1^{1}B_{u}$ whose polaronic feature is suggested to be 
responsible for the charge-transfer behavior, is estimated to lie below the $2^{1}A_{g}$ with a small gap of 0.7eV. For a more accurate 
estimate of the energy order and energy gaps, including the dynamic correlation can be considered as an effective way. %need to say something more...

%In this work, we introduce an algorithm which generally applies to any order of (transition) particle density matrices for DMRG wave functions. We first give a brief review on the DMRG method in Sec. Further we present a general algorithm to evaluate any order particle density matrices in Sec.. We show the application of this algorithm in the DMRG-NEVPT2 method, to study the static and dynamic correlation effects of photovoltaic polymer PPV in Sec.